Autocomplete script for TeXworks

The script is demonstrated by letting it make suggestions
for a sequence of usecases.
The demonstrated autocompletion behaviour is triggered by the
short cut "<crtl> + m", which are the only keys that are 
pressed during recording of this video.












Case 1:
=======
Fill in the rest of an incomplete word. The word must be 
present in one of the files currently opened by TeXworks.

Te
TeXworks












Case 2:
=======
If there are more than a single word that can be used for 
autocompletion in the opened documents, it is possible to 
cycle through all the possibe combinations by activating 
the script repeatedly.

i
incomplete










Case 3:
=======
Consider a document with the three long words
veryLongWord
veryLongWordWithCAPS
veryLongWordWithoutCAPS
and the current word that should be autocompleted is "ve".
Then will the script detect that there are three 
possibilities and that they all have a common prefix 
"veryLongWord". This common prefix is typed in and then the 
script is ready to await further input.

ve
veryLongWordWithoutCAPS





Case 4: 
=======
Filename completion based on files in the current directory
(the directory contains the following: 
autocompleteexample.tex
testing.log
testing.tex
subdir)

\input{}
hejmeddig

hejsa


hejsa


\begin{document}

\begin{figure}
\begin{center}
\begin{figure}
\begin{center}

\includegraphics{testing.tex}

hej

\end{center}
\end{figure}
\end{center}
\end{figure}
\end{document}




extractedWord
commandName
lastGuess
wordStart










Case 5: 
=======
Filename completion within subdirs.
The 'subdir' directory contains the two files
image.png and imageTwo.jpg.

(The forward slash '/' is provided by the keyboard)

\includegraphics{}
\includegraphics{subdir/imageTwo.jpg}









Case 6: 
=======
Closing of unclosed environments.

\begin{document}
\begin{table}
\begin{tabular}{cc}

\end{tabular}
\end{table}
\end{document}









